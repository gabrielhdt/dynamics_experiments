\documentclass{article}
\usepackage{typearea}
\usepackage{amsmath,amssymb,amsthm}
\usepackage{unicode-math}
\usepackage{fontspec}
\usepackage[frenchb]{babel}
\usepackage{ellipsis,xspace,xfrac}
\usepackage{tikz,subfig,graphicx}

\setmainfont[Ligatures=TeX]{Linux Libertine O}
\newcommand\cad{c'est-à-dire\xspace}
\newcommand\resp{resp.\xspace}
\newcommand\ssi{si et seulement si\xspace}
\newcommand\ie{i.e.\xspace}
\newcommand\eg{e.g.\xspace}

\theoremstyle{definition}
\newtheorem{defn}{Définition}

\theoremstyle{plain}
\newtheorem{prop}{Propriété}

\theoremstyle{remark}
\newtheorem*{rem}{Remarque}
\newtheorem{ex}{Exemple}

\title{Stabilité des systèmes dynamiques discrets}
\author{Gabriel \bsc{Hondet}}
\date{\today}

\begin{document}
\maketitle
\tableofcontents

\part{Solution particulière}
\section{Point particulier}
\subsection{Point fixe}
\subsection{Point périodique}
\subsection{Point critique}
\begin{defn}[point critique]
    Soit $f:\mathbb{R}\mapsto \mathbb{R}$,\\
    $x$ point critique \ssi $f'(x) = 0$.
\end{defn}
\section{Dérivée Schwarzienne}

\part{Stabilité structurelle}
\section{Bifurcation}
Stabilité de la fonction selon un paramètre.
\begin{ex}\label{ex:bif_pol_deg3}
    Soit $P_\mu(x) = \mu x(1-x^2)$
    \begin{figure}
        \centering
        \subfloat[$\mu = 1/2$]
        {
            \begin{tikzpicture}[domain=-1:1,scale=1.5]
                \draw [->] (0,-1) -- (0,1) node[above] {$y$};
                \draw [->] (-1,0) -- (1,0) node[right] {$x$};
                \draw[color=blue] plot[id=lin] function{x};
                \draw[color=orange] plot[id=deg3] function{0.5*x*(1-x**2)};
            \end{tikzpicture}\label{sfig:bif_mu0.5}
        }
        \quad
        \subfloat[$\mu = 1$]
        {
            \begin{tikzpicture}[domain=-1:1,scale=1.5]
               \draw [->] (0,-1) -- (0,1) node[above] {$y$};
               \draw [->] (-1,0) -- (1,0) node[right] {$x$};
               \draw[color=blue] plot[id=lin] function{x};
               \draw[color=orange] plot[id=deg3] function{x*(1-x**2)};
            \end{tikzpicture}\label{sfig:bif_mu1}
        }
        \quad
        \subfloat[$\mu = 2$]
        {
            \begin{tikzpicture}[domain=-1:1,scale=1.5]
               \draw [->] (0,-1) -- (0,1) node[above] {$y$};
               \draw [->] (-1,0) -- (1,0) node[right] {$x$};
               \draw[color=blue] plot[id=lin] function{x};
               \draw[color=orange] plot[id=deg3] function{2*x*(1-x**2)};
            \end{tikzpicture}\label{sfig:bif_mu2}
        }
        \caption{Bifurcation de la fonction $P_\mu(x)$ de l'exemple~\ref{ex:bif_pol_deg3}}\label{fig:bif_deg3}
    \end{figure}
\end{ex}
\section{C-distance}


\end{document}
