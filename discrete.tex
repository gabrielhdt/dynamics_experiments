\documentclass{article}
\usepackage{typearea}
\usepackage{amsmath,amssymb,amsthm}
\usepackage{unicode-math}
\usepackage{fontspec}
\usepackage[frenchb]{babel}
\usepackage{ellipsis,xspace,xfrac}
\usepackage{tikz,subfig,graphicx}

\setmainfont[Ligatures=TeX]{Linux Libertine O}
\addfontfeature{Fractions=On}
\newcommand\cad{c'est-à-dire\xspace}
\newcommand\resp{resp.\xspace}
\newcommand\ssi{si et seulement si\xspace}
\newcommand\ie{i.e.\xspace}
\newcommand\eg{e.g.\xspace}

\theoremstyle{definition}
\newtheorem{defn}{Définition}

\theoremstyle{plain}
\newtheorem{thm}{Théorème}
\newtheorem{prop}{Propriété}

\theoremstyle{remark}
\newtheorem*{rem}{Remarque}
\newtheorem{ex}{Exemple}

\title{Stabilité des systèmes dynamiques discrets}
\author{Gabriel \bsc{Hondet}}
\date{\today}

\begin{document}
\maketitle
\tableofcontents

\part{Solution particulière}
\section{Point particulier}
\subsection{Point fixe}
\subsection{Point périodique}
\subsection{Point critique}
\begin{defn}[point critique]
    Soit $f:\mathbb{R}\mapsto \mathbb{R}$,\\
    $x$ point critique \ssi $f'(x) = 0$.
\end{defn}
\section{Dérivée Schwarzienne}

\part{Stabilité structurelle}
\section{Bifurcation}
Stabilité de la fonction selon un paramètre.
\begin{thm}[The saddle-node bifurcation]
    Supposons
    \begin{enumerate}
        \item $f_{\lambda_0}(0) = 0$
        \item $f'_{\lambda_0}(0) = 0$
        \item $f''_{\lambda_0}(0) \neq 0$
        \item $\displaystyle{\frac{\partial f_\lambda}{\partial \lambda}
            \Bigg\rvert_{\lambda = \lambda_0}(0)
            \neq 0}$.
    \end{enumerate}
    Alors il existe un intervalle I autour de $0$ et une fonction douce
    $p:I\mapsto \mathbb{R}$ satisfaisant $p(0) = \lambda_0$ et telle que
    \begin{equation*}
        f_{p(x)}(x) = x.
    \end{equation*}
    De plus, $p'(0) = 0$ et $p''(0) \neq 0$.
\end{thm}
\begin{ex}\label{ex:bif_pol_deg3}
    Soit $P_\mu(x) = \mu x(1-x^2)$
    \begin{figure}
        \centering
        \subfloat[$\mu = 1/2$]
        {
            \begin{tikzpicture}[domain=-1:1,scale=1.5]
                \draw [->] (0,-1) -- (0,1) node[above] {$y$};
                \draw [->] (-1,0) -- (1,0) node[right] {$x$};
                \draw[color=blue] plot[id=lin] function{x};
                \draw[color=orange] plot[id=deg3] function{0.5*x*(1-x**2)};
            \end{tikzpicture}\label{sfig:bif_mu0.5}
        }
        \quad
        \subfloat[$\mu = 1$]
        {
            \begin{tikzpicture}[domain=-1:1,scale=1.5]
               \draw [->] (0,-1) -- (0,1) node[above] {$y$};
               \draw [->] (-1,0) -- (1,0) node[right] {$x$};
               \draw[color=blue] plot[id=lin] function{x};
               \draw[color=orange] plot[id=deg3] function{x*(1-x**2)};
            \end{tikzpicture}\label{sfig:bif_mu1}
        }
        \quad
        \subfloat[$\mu = 2$]
        {
            \begin{tikzpicture}[domain=-1:1,scale=1.5]
               \draw [->] (0,-1) -- (0,1) node[above] {$y$};
               \draw [->] (-1,0) -- (1,0) node[right] {$x$};
               \draw[color=blue] plot[id=lin] function{x};
               \draw[color=orange] plot[id=deg3] function{2*x*(1-x**2)};
            \end{tikzpicture}\label{sfig:bif_mu2}
        }
        \caption{Bifurcation de la fonction $P_\mu(x)$ de l'exemple~\ref{ex:bif_pol_deg3}}\label{fig:bif_deg3}
    \end{figure}
\end{ex}

\begin{thm}[Bifurcation de doublement de période]
    Supposons
    \begin{enumerate}
        \item $f_\lambda (0) = 0$ pour tout $\lambda$ dans un intervalle
            autour de $\lambda_0$.
        \item $f'_{\lambda_0} = -1$.
        \item $\displaystyle{\frac{\partial (f^2_\lambda)'}{\partial \lambda}
                \Bigg\rvert_{\lambda = \lambda_0} (0) \neq 0}$.
    \end{enumerate}
    Alors il existe un intervalle I autour de 0 et une fonction $p:I\mapsto\mathbb{R}$
    telle que \[f_{p(x)}(x) \neq x\] mais \[f^2_{p(x)}(x) = x.\]
\end{thm}

\begin{ex}\label{ex:bif_sinh}Soit $H_\mu(x) = \mu \sinh(x)$\\
    On a :
    \begin{enumerate}
        \item $H_\mu'(x) = -\mu\cosh(x)$
        \item $\displaystyle{\frac{\partial (f^2_\lambda)'}{\partial \lambda}
            \Bigg\rvert_{\lambda = \lambda_0} (0) =
            \cosh(x)[2\mu\cosh(\mu\sinh(x))
            + \mu^2\sinh(x)\sinh(\mu\sinh(x))]}$.
    \end{enumerate}
    On a donc:
    \begin{enumerate}
        \item $H_1(0) = 0$;
        \item $H_1'(0) = -1$;
        \item $\displaystyle{\frac{\partial (f^2_\mu)'}{\partial \mu}
            \Bigg\rvert_{\mu = \mu_0} (0) = 2}$.
    \end{enumerate}
    La fonction $H_\mu$ présente donc une bifurcation de doublement de
    période en $\mu = 1$.
\end{ex}
\begin{figure}
    \centering
    \subfloat[$\mu = 1/2$]{
        \begin{tikzpicture}[domain=-1:1,scale=1.5]
            \draw [->] (0,-1) -- (0,1) node[above] {$y$};
            \draw [->] (-1,0) -- (1,0) node[right] {$x$};
            \draw[color=blue] plot[id=lin] function{x};
            \draw[color=orange] plot[id=ex21] function{0.5*sinh(x)};
        \end{tikzpicture}\label{sfig:bif_ex2_0.5}
    }
    \quad
    \subfloat[$\mu = 1$]{
        \begin{tikzpicture}[domain=-1:1,scale=1.5]
            \draw [->] (0,-1) -- (0,1) node[above] {$y$};
            \draw [->] (-1,0) -- (1,0) node[right] {$x$};
            \draw[color=blue] plot[id=lin] function{x};
            \draw[color=orange] plot[id=ex22] function{sinh(x)};
        \end{tikzpicture}\label{sfig:bif_ex2_1}
    }
    \quad
    \subfloat[$\mu = 3/2$]{
        \begin{tikzpicture}[domain=-1:1,scale=1.5]
            \draw [->] (0,-1) -- (0,1) node[above] {$y$};
            \draw [->] (-1,0) -- (1,0) node[right] {$x$};
            \draw[color=blue] plot[id=lin] function{x};
            \draw[color=orange] plot[id=ex23] function{1.5*sinh(x)};
        \end{tikzpicture}\label{sfig:bif_ex2_1.5}
    }
    \caption{Variations du paramètre de la fonction $H_\mu$ de
        l'exemple~\ref{ex:bif_sinh}}
\end{figure}


\section{C-distance}


\end{document}
