\documentclass{article}
\usepackage{typearea}
\usepackage{amsmath,amssymb,amsthm}
\usepackage{unicode-math}
\usepackage{fontspec}
\usepackage[frenchb]{babel}
\usepackage{ellipsis,xspace,xfrac}

\setmainfont[Ligatures=TeX]{Linux Libertine O}
\newcommand\cad{c'est-à-dire\xspace}
\newcommand\resp{resp.\xspace}
\newcommand\ssi{si et seulement si\xspace}
\newcommand\ie{i.e.\xspace}
\newcommand\eg{e.g.\xspace}

\theoremstyle{definition}
\newtheorem{defn}{Définition}

\theoremstyle{plain}
\newtheorem{prop}{Propriété}

\theoremstyle{remark}
\newtheorem*{rem}{Remarque}
\newtheorem{ex}{Exemple}

\title{Stabilité des systèmes dynamiques discrets}
\author{Gabriel \bsc{Hondet}}
\date{\today}

\begin{document}
\maketitle
\tableofcontents

\part{Solution particulière}
\section{Point critique}
\begin{defn}[point critique]
    Soit $f:\mathbb{R}\mapsto \mathbb{R}$,\\
    $x$ point critique \ssi $f'(x) = 0$
\end{defn}
\section{Dérivée Schwarzienne}

\part{Stabilité structurelle}
\section{Bifurcation}
\section{C-distance}


\end{document}
